%%%%%%%%%%%%%%%%%%%%%%%%%%%%%%%%%%%%%%%%%
% Arsclassica Article
% LaTeX Template
% Version 1.1 (1/8/17)
%
% This template has been downloaded from:
% http://www.LaTeXTemplates.com
%
% Original author:
% Lorenzo Pantieri (http://www.lorenzopantieri.net) with extensive modifications by:
% Vel (vel@latextemplates.com)
%
% License:
% CC BY-NC-SA 3.0 (http://creativecommons.org/licenses/by-nc-sa/3.0/)
%
%%%%%%%%%%%%%%%%%%%%%%%%%%%%%%%%%%%%%%%%%

%----------------------------------------------------------------------------------------
%	PACKAGES AND OTHER DOCUMENT CONFIGURATIONS
%----------------------------------------------------------------------------------------

\documentclass[
10pt, % Main document font size
a4paper, % Paper type, use 'letterpaper' for US Letter paper
oneside, % One page layout (no page indentation)
%twoside, % Two page layout (page indentation for binding and different headers)
headinclude,footinclude, % Extra spacing for the header and footer
BCOR5mm, % Binding correction
]{scrartcl}

\usepackage[utf8]{inputenc}
\usepackage[spanish]{babel}
\usepackage{listings}


\input{structure.tex} % Include the structure.tex file which specified the document structure and layout

\hyphenation{Fortran hy-phen-ation} % Specify custom hyphenation points in words with dashes where you would like hyphenation to occur, or alternatively, don't put any dashes in a word to stop hyphenation altogether

%----------------------------------------------------------------------------------------
%	TITLE AND AUTHOR(S)
%----------------------------------------------------------------------------------------

\title{\normalfont\spacedallcaps{Práctica de tienda}} % The article title

\subtitle{Algoritmos y Estructuras de Datos I} % Uncomment to display a subtitle

\author{\spacedlowsmallcaps{Carlos Cañellas Tovar\textsuperscript{1} y Samuel Palazón Sandoval}} % The article author(s) - author affiliations need to be specified in the AUTHOR AFFILIATIONS block

\date{7 de diciembre de 2019} % An optional date to appear under the author(s)

%----------------------------------------------------------------------------------------

\begin{document}

%----------------------------------------------------------------------------------------
%	HEADERS
%----------------------------------------------------------------------------------------

\renewcommand{\sectionmark}[1]{\markright{\spacedlowsmallcaps{#1}}} % The header for all pages (oneside) or for even pages (twoside)
%\renewcommand{\subsectionmark}[1]{\markright{\thesubsection~#1}} % Uncomment when using the twoside option - this modifies the header on odd pages
\lehead{\mbox{\llap{\small\thepage\kern1em\color{halfgray} \vline}\color{halfgray}\hspace{0.5em}\rightmark\hfil}} % The header style

\pagestyle{scrheadings} % Enable the headers specified in this block

%----------------------------------------------------------------------------------------
%	TABLE OF CONTENTS & LISTS OF FIGURES AND TABLES
%----------------------------------------------------------------------------------------

\maketitle % Print the title/author/date block

\setcounter{tocdepth}{2} % Set the depth of the table of contents to show sections and subsections only

\tableofcontents % Print the table of contents

\listoffigures % Print the list of figures

\listoftables % Print the list of tables

%----------------------------------------------------------------------------------------
%	ABSTRACT
%----------------------------------------------------------------------------------------

% This section will not appear in the table of contents due to the star (\section*)
\section*{Preámbulo} 

Este documento es el resultado de muchas horas y otros cafés más. El esfuerzo realizado, la investigación de aquello que hemos necesitado, las decisiones tomadas para el diseño e implementación de este programa y su estructuración están todas plasmadas aquí.

Este documento ha sido creado usando \LaTeX, a partir de una plantilla que se adaptó a nuestras necesidades. 

%----------------------------------------------------------------------------------------
%	AUTHOR AFFILIATIONS
%----------------------------------------------------------------------------------------

\let\thefootnote\relax\footnotetext{\textsuperscript{1} \textit{Cuenta A18, usada en el juez online Mooshak.}}

%----------------------------------------------------------------------------------------

\newpage % Start the article content on the second page, remove this if you have a longer abstract that goes onto the second page

%----------------------------------------------------------------------------------------
%	INTRODUCTION
%----------------------------------------------------------------------------------------

\section{Introducción}

%Aquí hablaré de usar el algoritmo FNV-1 \cite{Noll:1994dg}.
Queda patente que es necesario crear un programa que, al ser ejecutado, nos ofrezca un resultado correcto en un tiempo prudencialmente rápido.

Para las necesidades de este programa, y dada la potencia actual de los ordenadores contemporáneos, al desarrollar este programa a bajo nivel, usando \textbf{C++}, nos encontramos con una respuesta (casi) inmediata. Se prooverá del tiempo usado por el programa en varias fases. Para las mediciones se usarán varios ordenadores con sistemas UNIX o UNIX-like, ya sea macOS o Ubuntu.
 
%----------------------------------------------------------------------------------------
%	ANÁLISIS Y DISEÑO DEL PROBLEMA
%----------------------------------------------------------------------------------------

\section{Análisis y diseño del problema}

\subsection{Ejercicio 006 - Diccionario de productos}

A partir del ejercicio \textbf{006} del juez online \textbf{Mooshak}, se empezó a tomar decisiones. ¿Incluímos el intérprete solamente separado? ¿Lo convertimos en clase? ¿Métodos estáticos o dinámicos?

Así, la primera decisión tomada fue crear una clase \textit{Intérprete} que contenga los métodos en el objeto creado (no estáticos).

El diccionario de productos es una clase que contiene una lista de productos, un contador del nº de productos (aunque hay que tener en cuenta que esta variable puede no ser necesaria en una tienda de verdad, pues se puede calcular este número), una tabla de dispersión (ver próxima sección) y un árbol de precios (ver la siguiente)

\subsection{Ejercicio 201 - Tabla de dispersión de palabras}


Tras ello, en el ejercicio \textbf{201} se ha decidido utilizar tablas de dispersión abiertas de tamaño variable.

El tiempo que usa la aplicación en macOS Catalina:

\begin{lstlisting}
% time ./a.out < IO/201a.in > salida 
0,11s user 0,08s system 97% cpu 0,195 total
\end{lstlisting}



\subsection{Ejercicio 301 - Árbol de precios}

Para realizar el ejercicio, había que decidir qué árbol usar. 

\begin{enumerate}[noitemsep] % [noitemsep] removes whitespace between the items for a compact look
\item Árbol trie: descartado, sería más útil para un árbol de productos.
\item Árbol B: planteable, pero los precios van con mínimo y máximo, por tanto...
\item Árbol binario AVL: la mejor opción, permite organizar precios de menor a mayor de forma equilibrada.
\end{enumerate}


\lipsum[6] % Dummy text

\paragraph{Paragraph Description} \lipsum[7] % Dummy text

\paragraph{Different Paragraph Description} \lipsum[8] % Dummy text

%------------------------------------------------

\subsection{Math}

\lipsum[4] % Dummy text

\begin{equation}
\cos^3 \theta =\frac{1}{4}\cos\theta+\frac{3}{4}\cos 3\theta
\label{eq:refname2}
\end{equation}

\lipsum[5] % Dummy text

\begin{definition}[Gauss] 
To a mathematician it is obvious that
$\int_{-\infty}^{+\infty}
e^{-x^2}\,dx=\sqrt{\pi}$. 
\end{definition} 

\begin{theorem}[Pythagoras]
The square of the hypotenuse (the side opposite the right angle) is equal to the sum of the squares of the other two sides.
\end{theorem}

\begin{proof} 
We have that $\log(1)^2 = 2\log(1)$.
But we also have that $\log(-1)^2=\log(1)=0$.
Then $2\log(-1)=0$, from which the proof.
\end{proof}

%----------------------------------------------------------------------------------------
%	RESULTS AND DISCUSSION
%----------------------------------------------------------------------------------------

\section{Results and Discussion}

Reference to Figure~\vref{fig:gallery}. % The \vref command specifies the location of the reference

\begin{figure}[tb]
\centering 
\includegraphics[width=0.5\columnwidth]{GalleriaStampe} 
\caption[An example of a floating figure]{An example of a floating figure (a reproduction from the \emph{Gallery of prints}, M.~Escher,\index{Escher, M.~C.} from \url{http://www.mcescher.com/}).} % The text in the square bracket is the caption for the list of figures while the text in the curly brackets is the figure caption
\label{fig:gallery} 
\end{figure}

\lipsum[10] % Dummy text

%------------------------------------------------

\subsection{Subsection}

\lipsum[11] % Dummy text

\subsubsection{Subsubsection}

\lipsum[12] % Dummy text

\begin{description}
\item[Word] Definition
\item[Concept] Explanation
\item[Idea] Text
\end{description}

\lipsum[12] % Dummy text

\begin{itemize}[noitemsep] % [noitemsep] removes whitespace between the items for a compact look
\item First item in a list
\item Second item in a list
\item Third item in a list
\end{itemize}

\subsubsection{Table}

\lipsum[13] % Dummy text

\begin{table}[hbt]
\caption{Table of Grades}
\centering
\begin{tabular}{llr}
\toprule
\multicolumn{2}{c}{Name} \\
\cmidrule(r){1-2}
First name & Last Name & Grade \\
\midrule
John & Doe & $7.5$ \\
Richard & Miles & $2$ \\
\bottomrule
\end{tabular}
\label{tab:label}
\end{table}

Reference to Table~\vref{tab:label}. % The \vref command specifies the location of the reference

%------------------------------------------------

\subsection{Figure Composed of Subfigures}

Reference the figure composed of multiple subfigures as Figure~\vref{fig:esempio}. Reference one of the subfigures as Figure~\vref{fig:ipsum}. % The \vref command specifies the location of the reference

\lipsum[15-18] % Dummy text

\begin{figure}[tb]
\centering
\subfloat[A city market.]{\includegraphics[width=.45\columnwidth]{Lorem}} \quad
\subfloat[Forest landscape.]{\includegraphics[width=.45\columnwidth]{Ipsum}\label{fig:ipsum}} \\
\subfloat[Mountain landscape.]{\includegraphics[width=.45\columnwidth]{Dolor}} \quad
\subfloat[A tile decoration.]{\includegraphics[width=.45\columnwidth]{Sit}}
\caption[A number of pictures.]{A number of pictures with no common theme.} % The text in the square bracket is the caption for the list of figures while the text in the curly brackets is the figure caption
\label{fig:esempio}
\end{figure}

%----------------------------------------------------------------------------------------
%	BIBLIOGRAPHY
%----------------------------------------------------------------------------------------

\renewcommand{\refname}{\spacedlowsmallcaps{Referencias}} % For modifying the bibliography heading

\bibliographystyle{unsrt}

\bibliography{sample.bib} % The file containing the bibliography

%----------------------------------------------------------------------------------------

\end{document}